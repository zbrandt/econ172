\documentclass{article}

\usepackage{fancyhdr}
\usepackage{extramarks}
\usepackage{amsmath}
\usepackage{amsthm}
\usepackage{amsfonts}
\usepackage{tikz}
\usepackage[plain]{algorithm}
\usepackage{algpseudocode}
\usepackage{hyperref}
\usepackage{booktabs}
\usepackage{parskip}

\usetikzlibrary{automata,positioning}

%
% Basic Document Settings
%

\topmargin=-0.45in
\evensidemargin=0in
\oddsidemargin=0in
\textwidth=6.5in
\textheight=9.0in
\headsep=0.25in

\linespread{1.1}

\pagestyle{fancy}
\lhead{\hmwkAuthorName}
\chead{\hmwkClass: \hmwkTitle}
\rhead{\firstxmark}
\lfoot{\lastxmark}
\cfoot{\thepage}

\renewcommand\headrulewidth{0.4pt}
\renewcommand\footrulewidth{0.4pt}

\setlength\parindent{0pt}

\hypersetup{
  colorlinks=true,
  urlcolor=blue
}%

% Create Problem Sections
%

\newcommand{\enterProblemHeader}[1]{
    \nobreak\extramarks{}{Problem \arabic{#1} continued on next page\ldots}\nobreak{}
    \nobreak\extramarks{Problem \arabic{#1} (continued)}{Problem \arabic{#1} continued on next page\ldots}\nobreak{}
}

\newcommand{\exitProblemHeader}[1]{
    \nobreak\extramarks{Problem \arabic{#1} (continued)}{Problem \arabic{#1} continued on next page\ldots}\nobreak{}
    \stepcounter{#1}
    \nobreak\extramarks{Problem \arabic{#1}}{}\nobreak{}
}

\setcounter{secnumdepth}{0}
\newcounter{partCounter}
\newcounter{homeworkProblemCounter}
\setcounter{homeworkProblemCounter}{1}
\nobreak\extramarks{Problem \arabic{homeworkProblemCounter}}{}\nobreak{}

%
% Homework Problem Environment
%
% This environment takes an optional argument. When given, it will adjust the
% problem counter. This is useful for when the problems given for your
% assignment aren't sequential. See the last 3 problems of this template for an
% example.
%
\newenvironment{homeworkProblem}[2][-1]{
    \ifnum#1>0
        \setcounter{homeworkProblemCounter}{#1}
    \fi
    \section{Problem \arabic{homeworkProblemCounter}: #2}
    \setcounter{partCounter}{1}
    \enterProblemHeader{homeworkProblemCounter}
}{
    \exitProblemHeader{homeworkProblemCounter}
}

%
% Homework Details
%   - Title
%   - Due date
%   - Class
%   - Section/Time
%   - Instructor
%   - Author
%

\newcommand{\hmwkTitle}{Problem Set\ \#2}
\newcommand{\hmwkDueDate}{October 20, 2025}
\newcommand{\hmwkClass}{Economics 172}
\newcommand{\hmwkClassTime}{}
\newcommand{\hmwkClassInstructor}{}
\newcommand{\hmwkAuthorName}{\textbf{Zachary Brandt}}
\newcommand{\hmwkAuthorEmail}{\href{mailto:zbrandt@berkeley.edu}{zbrandt@berkeley.edu}}

%
% Title Page
%

\title{
    \vspace{2in}
    \textmd{\textbf{\hmwkClass:\ \hmwkTitle}}\\
    \normalsize\vspace{0.1in}\small{Due\ on\ \hmwkDueDate\ at 11:00pm}\\
    \vspace{0.1in}\large{\textit{\hmwkClassInstructor\ \hmwkClassTime}}
    \vspace{3in}
}

\author{\hmwkAuthorName \\ \hmwkAuthorEmail}
\date{}

\renewcommand{\part}[1]{\textbf{\large Part \Alph{partCounter}}\stepcounter{partCounter}\\}

%
% Various Helper Commands
%

% Useful for algorithms
\newcommand{\alg}[1]{\textsc{\bfseries \footnotesize #1}}

% For derivatives
\newcommand{\deriv}[1]{\frac{\mathrm{d}}{\mathrm{d}x} (#1)}

% For partial derivatives
\newcommand{\pderiv}[2]{\frac{\partial}{\partial #1} (#2)}

% Integral dx
\newcommand{\dx}{\mathrm{d}x}

% Homework Solution Environment
\newenvironment{solution}[1][\large Solution]{
  \color{blue}
  \smallskip
  \textbf{#1}
}{}
% Probability commands: Expectation, Variance, Covariance, Bias
\newcommand{\E}{\mathrm{E}}
\newcommand{\Var}{\mathrm{Var}}
\newcommand{\Cov}{\mathrm{Cov}}
\newcommand{\Bias}{\mathrm{Bias}}

\begin{document}

\maketitle

\pagebreak

\begin{homeworkProblem}{Short questions}

    \begin{enumerate}
        \item What is the poverty gap? What is the drawback of poverty head 
        count ratio. What is the advantage of Poverty gap over the headcount 
        ratio?
        \item Consider a country in which 20 percent of the population live on 
        20 cents per day, 15 percent live on 30 cents per day, 15 percent live 
        on 50 cents per day, 10 percent live on 70 cents per day, 20 percent 
        live on 90 cents per day, and 20 percent live on 1.50 dollars per day. 
        If the poverty line is set at one dollar per day, what is the poverty 
        headcount ratio? What is the poverty gap? What is the poverty severity?
        \item Show an example that $P_1$ (poverty gap) and P2 (poverty 
        severity) can change while $P_0$ (headcount ratio) is fixed. (You may 
        not need to calculate each value)
        \item How $P_0$ (headcount ratio), $P_1$ (poverty gap), $P_2$ (poverty 
        severity) change in graph (a), (b), (c) between situation 1 (black 
        line) and situation 2 (blue line)?
'    \end{enumerate}

\end{homeworkProblem}

\pagebreak

\begin{homeworkProblem}
    
    A group of researchers wishes to test whether cash transfers to low-income 
    households have any spillovers on local small business owners. Their 
    sample consists of 160 villages across 20 districts in India. The 
    researchers randomly assign 80 villages to control and 80 villages to 
    treatment, stratifying by district. In control villages, no households 
    receive cash transfers. In treatment villages, approximately one fourth 
    of households receive a cash transfer. \\

    The researchers conduct surveys with a sample of 10–15 small business 
    owners in each village 3 months after the cash transfers are disbursed. 
    Note that none of the small business owners are also cash transfer 
    recipients. In these surveys, the researchers collect information such 
    as the nature of the business as well as monthly revenues, labor costs, 
    input costs, and profits. The researchers run the following regression:

    \[
        Y_{ivd} = \alpha_0 + \alpha_1 \text{Treatment}_{v} + \epsilon_{ivd},
    \]

    where $Y_{ivd}$ represents outcomes of interest (revenues, profits) for 
    business owner $i$ in village $v$ in district $d$, and Treatmentv denotes 
    treatment status at the village level. \\

    Here is the regression table and results (standard errors in parentheses). 
    The “Village wealth index” (VWI) combines several indicators of village 
    wealth, constructed so the index has mean zero and standard deviation one. 
    Positive values indicate a wealthier village, and the index is measured in 
    standard deviation units. “Above median business assets” is a dummy 
    variable equal to 1 if the business has above median assets and 0 
    otherwise.

    \begin{itemize}
        \item[(a)] Write out the regression equation for results presented in 
        column (3). What’s the variable of interest and coefficient?
        \item[(b)] Columns (1) and (2) above show the results of this analysis. 
        Interpret these results, commenting on the magnitude and significance 
        of the parameters of interest.
        \item[(c)] What is the 95\% confidence interval around the estimated 
        treatment effect in column (1) and (2)? What does this confidence 
        interval tell you?
        \item[(d)] In columns (3) and (4), the researchers test whether the 
        cash transfers had differential impacts on business revenues based on 
        individual business owner’s characteristics and village characteristics. 
        Write down the regressions depicted in columns (3) and (4). Interpret 
        each of the parameters in detail.
        \item[(e)] How would you summarize these results?
    \end{itemize}

\end{homeworkProblem}

\pagebreak

\end{document}