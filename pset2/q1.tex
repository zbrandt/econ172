\begin{homeworkProblem}{RCT}

    This question uses an adapted dataset based on Muralidharan, Singh, and 
    Ganimian’s (2019) paper \textit{Disrupting Education? Experimental 
    Evidence on Technology-Aided Instruction in India}. The paper is available 
    online and the replication data is available on ICPSR. Download the 
    adapted dataset from bCourses.

    This project evaluated the impact of a center-based and technology-aided 
    after-school educational program on math and Hindi performance among middle
    schoolers living in low-income neighborhoods in urban India. The 
    technology-based curriculum was designed to be high-quality, adaptive, 
    and engaging. Approximately 600 middle schoolers were recruited to 
    participate in the study. Half of these recruited students were randomly
    allocated by lottery to receive a voucher to participate in the program 
    (treatment group), and half were not (control group). For the purposes of 
    this question, you can assume that 100\% of those assigned to treatment 
    participated in the program and that 0\% of those assigned to control 
    participated in the program.

    Below is a list of the variables included in the dataset, with a brief 
    description of each. Note that BL refers to versions of variables 
    collected at baseline (collected before the program began) while EL refers 
    to variables collected at endline (collected at the conclusion of the 
    program).

    \begin{table}[h!]
        \centering
        \begin{tabular}{ll}
            \toprule
            \textbf{Variable} & \textbf{Description} \\
            \midrule
            student\_id & Identification numbers that uniquely identify students \\
            student\_age & Age of the student (collected at baseline) \\
            student\_female & Indicator (0/1) for whether the student is female \\
            student\_grade & Grade of the student (collected at baseline) \\
            treatment & Indicator (0/1) for treatment status of the student \\
            BL\_math\_percent, EL\_math\_percent & Math score in percent correct \\
            BL\_hindi\_percent, EL\_hindi\_percent & Hindi score in percent correct \\
            BL\_ses\_index, EL\_ses\_index & Household wealth index \\
            \bottomrule
        \end{tabular}
    \end{table}

    \begin{enumerate}
        \item Import the data and generate a table of summary statistics. What 
            is the range of ages and grade levels within this sample? What are 
            average scores for math and Hindi at baseline? At endline?

        \item Next you’ll check whether the treatment and control groups are 
            balanced in terms of each of the following variables: (1) age, (2) 
            sex, (3) household wealth index at baseline, (4) math scores at 
            baseline, and (5) Hindi scores at baseline. (Hint: To do this, 
            you’ll run a separate regression for each of these five variables. 
            Use \texttt{stargazer} or other preferable command to output the 
            tables.) Before you code up any regressions, write down the 
            regression you plan to run for at least one of the variables. 
            Which parameter represents the coefficient of interest? What do 
            you expect the estimate of this parameter to be (positive, 
            negative, zero) and why?

        \item Use the \texttt{lm} command to run the regressions in R. Are 
            there significant differences between the treatment and control 
            groups for any of these five variables? What is the purpose of 
            this exercise, and what are you able to conclude?

        \item Next, you’ll estimate the impact of the treatment on math and 
            Hindi scores at endline.

        \begin{enumerate}
            \item First, write the two regressions you will run to estimate 
                these treatment effects. In words, what will each of these 
                parameters capture?
            \item Run the two regressions using the \texttt{lm} command and 
                use the \texttt{stargazer} command to produce tables containing 
                the results of these two regressions. Interpret your results. 
                What is the effect of the treatment on each of math and Hindi 
                scores? Are these estimated treatment effects statistically 
                significant? Explain.
        \end{enumerate}
        
    \end{enumerate}

\end{homeworkProblem}