\begin{homeworkProblem}{RCT}

    This question uses an adapted dataset based on Muralidharan, Singh, and 
    Ganimian’s (2019) paper \textit{Disrupting Education? Experimental 
    Evidence on Technology-Aided Instruction in India}. The paper is available 
    online and the replication data is available on ICPSR. Download the 
    adapted dataset from bCourses.

    This project evaluated the impact of a center-based and technology-aided 
    after-school educational program on math and Hindi performance among middle
    schoolers living in low-income neighborhoods in urban India. The 
    technology-based curriculum was designed to be high-quality, adaptive, 
    and engaging. Approximately 600 middle schoolers were recruited to 
    participate in the study. Half of these recruited students were randomly
    allocated by lottery to receive a voucher to participate in the program 
    (treatment group), and half were not (control group). For the purposes of 
    this question, you can assume that 100\% of those assigned to treatment 
    participated in the program and that 0\% of those assigned to control 
    participated in the program.

    Below is a list of the variables included in the dataset, with a brief 
    description of each. Note that BL refers to versions of variables 
    collected at baseline (collected before the program began) while EL refers 
    to variables collected at endline (collected at the conclusion of the 
    program).

    \begin{table}[h!]
        \centering
        \begin{tabular}{ll}
            \toprule
            \textbf{Variable} & \textbf{Description} \\
            \midrule
            student\_id & Identification numbers that uniquely identify students \\
            student\_age & Age of the student (collected at baseline) \\
            student\_female & Indicator (0/1) for whether the student is female \\
            student\_grade & Grade of the student (collected at baseline) \\
            treatment & Indicator (0/1) for treatment status of the student \\
            BL\_math\_percent, EL\_math\_percent & Math score in percent correct \\
            BL\_hindi\_percent, EL\_hindi\_percent & Hindi score in percent correct \\
            BL\_ses\_index, EL\_ses\_index & Household wealth index \\
            \bottomrule
        \end{tabular}
    \end{table}

    \begin{enumerate}
        \item Import the data and generate a table of summary statistics. What 
            is the range of ages and grade levels within this sample? What are 
            average scores for math and Hindi at baseline? At endline?

        \item Next you’ll check whether the treatment and control groups are 
            balanced in terms of each of the following variables: (1) age, (2) 
            sex, (3) household wealth index at baseline, (4) math scores at 
            baseline, and (5) Hindi scores at baseline. (Hint: To do this, 
            you’ll run a separate regression for each of these five variables. 
            Use \texttt{stargazer} or other preferable command to output the 
            tables.) Before you code up any regressions, write down the 
            regression you plan to run for at least one of the variables. 
            Which parameter represents the coefficient of interest? What do 
            you expect the estimate of this parameter to be (positive, 
            negative, zero) and why?

        \item Use the \texttt{lm} command to run the regressions in R. Are 
            there significant differences between the treatment and control 
            groups for any of these five variables? What is the purpose of 
            this exercise, and what are you able to conclude?

        \item Next, you’ll estimate the impact of the treatment on math and 
            Hindi scores at endline.

        \begin{enumerate}
            \item First, write the two regressions you will run to estimate 
                these treatment effects. In words, what will each of these 
                parameters capture?
            \item Run the two regressions using the \texttt{lm} command and 
                use the \texttt{stargazer} command to produce tables containing 
                the results of these two regressions. Interpret your results. 
                What is the effect of the treatment on each of math and Hindi 
                scores? Are these estimated treatment effects statistically 
                significant? Explain.
        \end{enumerate}
        
    \end{enumerate}

    \begin{solution}
        
        \begin{enumerate}
            \item Below is a table of summary statistics I have generated
                for the data. 

                \begin{table}[!htbp] \centering \color{blue}
                    \resizebox{0.6\textwidth}{!}{%
                    \begin{tabular}{@{\extracolsep{5pt}}lccccc} 
                        \\[-1.8ex]\hline 
                        \hline \\[-1.8ex] 
                        Statistic & \multicolumn{1}{c}{N} & \multicolumn{1}{c}{Mean} & \multicolumn{1}{c}{St. Dev.} & \multicolumn{1}{c}{Min} & \multicolumn{1}{c}{Max} \\ 
                        \hline \\[-1.8ex] 
                        student\_id & 533 & 310.246 & 177.595 & 1 & 619 \\ 
                        treatment & 533 & 0.493 & 0.500 & 0 & 1 \\ 
                        student\_age & 533 & 12.413 & 1.357 & 10 & 15 \\ 
                        student\_female & 533 & 0.771 & 0.421 & 0 & 1 \\ 
                        student\_grade & 533 & 7.182 & 1.101 & 4 & 9 \\ 
                        BL\_math\_percent & 533 & 0.316 & 0.124 & 0.010 & 0.758 \\ 
                        BL\_hindi\_percent & 533 & 0.435 & 0.167 & 0.041 & 0.923 \\ 
                        BL\_ses\_index & 533 & $-$0.053 & 1.657 & $-$5.548 & 4.117 \\ 
                        EL\_math\_percent & 533 & 0.504 & 0.179 & $-$0.009 & 1.007 \\ 
                        EL\_hindi\_percent & 533 & 0.555 & 0.193 & 0.072 & 1.005 \\ 
                        EL\_ses\_index & 533 & $-$0.059 & 1.661 & $-$5.681 & 4.128 \\ 
                        \hline \\[-1.8ex] 
                    \end{tabular} 
                    }%
                \end{table}

                The range of ages within this sample is from 10 to 15 years 
                old. The range of grade levels within this sample is from 4 to 
                9. The average math score at baseline is 31.6\% correct, while 
                the average Hindi score at baseline is 43.5\% correct. At 
                endline, the average math score is 50.4\% correct and the 
                average Hindi score is 55.5\% correct.

            \item To check whether the treatment and control groups are
                balanced in terms of each of the variables age, sex, household
                wealth at baseline, math scores at baseline, and Hindi scores
                at baseline, I plan to run the following regression. 
                \[
                    \text{student\_age}_i = \alpha + \beta \cdot \text{treatment}_i + \epsilon_i
                \]

                where $\beta$ represents the coefficient of interest, the
                average difference in student\_age between the treatment and 
                control groups. I expect the estimate of this paramter to be 
                zero since the treatment and control groups should be balanced
                in terms of this variable and the four others mentioned. 

            \item Below is a regression table for the variables age, sex, 
                household wealth, and math and Hindi scores.

                \begin{table}[!htbp]
                    \centering
                    \color{blue}
                    \resizebox{1\textwidth}{!}{%
                        \begin{tabular}{@{}lccccc@{}}
                        \\[-1.8ex]\hline 
                        \hline \\[-1.8ex] 
                        & \multicolumn{5}{c}{\textit{Dependent variable:}} \\ 
                        \cline{2-6} 
                        \\[-1.8ex] & student\_age & student\_female & BL\_ses\_index & BL\_math\_percent & BL\_hindi\_percent \\ 
                        \\[-1.8ex] & (1) & (2) & (3) & (4) & (5)\\ 
                        \hline \\[-1.8ex] 
                        treatment & 0.153 & 0.001 & $-$0.191 & $-$0.014 & 0.010 \\ 
                        & (0.117) & (0.036) & (0.143) & (0.011) & (0.014) \\ 
                        & & & & & \\ 
                        Constant & 12.337$^{***}$ & 0.770$^{***}$ & 0.041 & 0.323$^{***}$ & 0.430$^{***}$ \\ 
                        & (0.083) & (0.026) & (0.101) & (0.008) & (0.010) \\ 
                        & & & & & \\ 
                        \hline \\[-1.8ex] 
                        Observations & 533 & 533 & 533 & 533 & 533 \\ 
                        R$^{2}$ & 0.003 & 0.00000 & 0.003 & 0.003 & 0.001 \\ 
                        Adjusted R$^{2}$ & 0.001 & $-$0.002 & 0.001 & 0.001 & $-$0.001 \\ 
                        Residual Std. Error (df = 531) & 1.356 & 0.421 & 1.656 & 0.124 & 0.167 \\ 
                        F Statistic (df = 1; 531) & 1.707 & 0.002 & 1.766 & 1.572 & 0.522 \\ 
                        \hline 
                        \hline \\[-1.8ex] 
                        \end{tabular}%
                    }
                \end{table}
                
                \pagebreak

                There are no significant differences between the treatment and
                control groups for any of these five variables, as all p-values
                are above 0.05. The purpose of this exercise is to verify that
                the random assignment of students to treatment and control groups
                was successful in creating balanced groups. Since there are no
                significant differences between the treatment and control groups
                for any of these five variables, I can conclude that the random
                assignment was successful.

            \item \begin{enumerate}
                \item To estimate the impact of the treatment on math and
                    Hindi scores, I will run the following two regressions.
                    \[
                        \begin{split}
                            \text{EL\_math\_percent}_i &= \alpha + \beta \cdot \text{treatment}_i + \epsilon_i \\
                            \text{EL\_hindi\_percent}_i &= \alpha + \beta \cdot \text{treatment}_i + \epsilon_i \\
                        \end{split}
                    \]

                    The coefficient of interest in both of these regressions 
                    is the parameter $\beta$, which captures the effect of the 
                    treatment on math and Hindi scores.
                \item Below is a regression table for the impact of the
                    treatment on math and Hindi scores.
                    
                    \begin{table}[!htbp] \centering \color{blue}
                    \resizebox{0.65\textwidth}{!}{%
                        \begin{tabular}{@{\extracolsep{5pt}}lcc} 
                        \\[-1.8ex]\hline 
                        \hline \\[-1.8ex] 
                        & \multicolumn{2}{c}{\textit{Dependent variable:}} \\ 
                        \cline{2-3} 
                        \\[-1.8ex] & EL\_math\_percent & EL\_hindi\_percent \\ 
                        \\[-1.8ex] & (1) & (2)\\ 
                        \hline \\[-1.8ex] 
                        treatment & 0.077$^{***}$ & 0.065$^{***}$ \\ 
                        & (0.015) & (0.016) \\ 
                        & & \\ 
                        Constant & 0.466$^{***}$ & 0.523$^{***}$ \\ 
                        & (0.011) & (0.012) \\ 
                        & & \\ 
                        \hline \\[-1.8ex] 
                        Observations & 533 & 533 \\ 
                        R$^{2}$ & 0.047 & 0.029 \\ 
                        Adjusted R$^{2}$ & 0.045 & 0.027 \\ 
                        Residual Std. Error (df = 531) & 0.175 & 0.190 \\ 
                        F Statistic (df = 1; 531) & 26.040$^{***}$ & 15.689$^{***}$ \\ 
                        \hline 
                        \hline \\[-1.8ex] 
                        \end{tabular} 
                        }%
                    \end{table} 

                    The effect of the treatment on math scores is an increase 
                    of 7.7 percentage points, which is statistically 
                    significant at the 1\% level. This suggests that the 
                    treatment had a positive impact on students' math scores.
                    The effect of the treatment on Hindi scores is an increase
                    of 6.5 percentage points, which is also statistically
                    significant at the 1\% level. This suggests that the
                    treatment had a positive impact on students' Hindi scores
                    as well.

            \end{enumerate}
                
        \end{enumerate}

    \end{solution}

\end{homeworkProblem}