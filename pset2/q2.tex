\begin{homeworkProblem}{Diff-in-Diff}
    
    We often study the effect of interventions on those who actually 
    received treatment. However, it is possible that economic 
    interventions can create positive externalities for non-recipient 
    households, for instance, through market interactions or financial 
    transfers between households. In this part of the problem set, we 
    will estimate the spillover effects of a fictitious cash transfer 
    program in a rural Tanzanian setting.

    Starting in 2015, imagine that thousands of households in rural 
    northern Tanzania received large cash transfers of roughly US 
    \$1,000 from the international non-governmental organization 
    GiveDirectly (GD). These cash transfers were targeted to relatively 
    poor households within certain villages but not others. The cash 
    grants were unconditional, and the amount is equivalent to more 
    than 50\% of total annual income for many recipient households. GD 
    expanded cash transfers in some villages and not others in a 
    non-randomized fashion. Then in 2018, approximately three years 
    after the distribution of the cash transfers, the households were 
    surveyed and measures were collected on many economic, social and 
    life outcomes, including household per capita income.

    The goal of this question is to estimate the impact of these 
    transfers on the households deemed to be too wealthy to be eligible 
    for GD transfers. Comparisons will be made across these ineligible 
    households in the "cash" villages (locations where poor households 
    received cash transfers from GD) versus in control villages (where 
    poor household did not receive GD transfers). Any effects on these 
    ineligible households can be considered spillovers since none of 
    them received cash transfers through the program.
        
    \begin{enumerate}
        \item Discuss the econometric assumptions needed to make a 
            difference-in-differences (DD) approach appropriate in the case 
            of understanding the spillover impacts of cash transfers on 
            household incomes.

        \item Why is it important that we have access to data from both 
            the baseline (pre-intervention) and endline (post-intervention) 
            survey rounds?

        \item How does the lack of randomization in the allocation of cash 
            transfers across villages affect the estimation of treatment 
            effects, and how might it lead to omitted variable bias?

        \item Please download the data to be used in the analysis. The 
            dataset \texttt{Ec172\_Fall25\_PS2Q2\_data.csv} is a partial 
            extract of project data (although the actual data has been 
            modified in various ways). Each observation (row) in the dataset 
            represents one household in one time period.

        All households in this problem set dataset are 
        \textbf{ineligible} for the GD cash transfers (i.e., they are 
        too wealthy to receive cash transfers). There are two 
        observations for each household, one from the baseline survey 
        (time=0) and one from the endline survey (time=1). Using the 
        \texttt{lm.cluster} command, determine the average difference at 
        baseline (time=0) between the households in villages where 
        eligible households received cash transfers (\texttt{cash}=1) 
        versus the households in control villages that did not receive 
        transfers (\texttt{cash}=0) for each of the following two 
        characteristics: household income per capita 
        (\texttt{income\_pc}) and if the household has an elderly member 
        (\texttt{elder}=1). Importantly, these characteristics were 
        collected by the survey team in the baseline survey before the 
        cash transfers were sent out. (Consider these two 
        characteristics one by one, that is, you should run two separate 
        regressions here.) Make sure to account for the correlation 
        among households in the same village by ``clustering'' your 
        standard errors by village (\texttt{village\_code}).

        \textbf{Hint:} To cluster SEs we can use the 
        \texttt{lm.cluster} command or the \texttt{feols} command. To use 
        the \texttt{lm.cluster} command in R, first install the packages 
        \texttt{miceadds} and \texttt{sandwich}, and then load them into 
        your library. The \texttt{lm.cluster} command functions just like 
        \texttt{lm}, but with the added option to specify a clustering 
        variable.

        For the \texttt{feols} command, install and load the 
        \texttt{fixest} package. Clustering is specified at the end of 
        the command using \texttt{, cluster ~ village\_code}. The syntax 
        of \texttt{feols} is otherwise similar to \texttt{lm}, except 
        that you do not include \texttt{formula =} before the regression 
        equation.

        \begin{enumerate}
            \item Report the regression output for the two regressions 
                mentioned above, and interpret the coefficients. Please 
                also discuss the standard errors and t-statistics. Taken 
                together, are ineligible households similar in cash 
                villages versus control villages along these two 
                dimensions? How does this finding inform the discussion of 
                the validity of the difference-in-differences (DD) 
                approach that you laid out above, if at all?
            \item Carry out a difference-in-differences (DD) analysis 
                estimating the spillover impact of GD cash transfers 
                among ineligible households, where the outcome of interest 
                is per capita income (\texttt{income\_pc}). You will need 
                to use data from both time periods. Once again, make sure 
                to account for the correlation among households in the 
                same village by clustering by village 
                (\texttt{village\_code}). 
                
                (Hint: recall that a DD analysis requires the inclusion 
                of separate explanatory variables for post 
                (\texttt{time}), treatment (\texttt{cash}), and an 
                interaction between the two variables. Note you will need 
                to create a new variable in your dataset for this 
                interaction term.) 
                
                What is the spillover impact among ineligible households 
                when poor households in their village receive large cash 
                transfers? Is this effect significantly different from 
                zero at 95\% confidence?
            \item Discuss at least two reasons why cash transfers 
                received by other households in one's village might 
                generate spillover effects over time even among 
                non-recipient households. Given these likely mechanisms, 
                does the magnitude of the estimated effect in part (b) 
                seem plausible? If not, why not?
        \end{enumerate}
    \end{enumerate}

    \begin{solution}
        \begin{enumerate}
            \item The econometric assumption needed to make a
                difference-in-differences approach appropriate in this case of
                estimating the spillover impact of cash transfers on household
                income is the parallel trends assumption. That is, in the 
                absence of the treatment, the treatment and control groups 
                experience the same trend in outcomes over time. In this case,
                that means that the ineligible households in ``cash'' villages
                and control villages would have followed the same trend in 
                household incomes if there had been no cash transfers to 
                eligible households in their respective villages. This also
                means that there are no other time-varying factors that 
                differentially affect the treatment and control groups over 
                time. However, time invariant differences between the groups 
                are controlled for since the difference-in-differences approach 
                will difference them out.
            \item Access to data from both the baseline and endline survey
                rounds is important because without baseline data, it is 
                impossible to estimate the difference-in-differences effect. 
                The treatment counterfactual trend is estimated using the 
                baseline data, which is subtracted from the post-treatment,
                endline data difference to isolate the treatment effect.
            \item While without randomization, a difference-in-differences 
                approach removes time-invariant omitted variable bias if the
                trends are in parallel, if there are differences 
                between the treatment and control group villages that determine
                different growth trajectories in household income over time, then
                the lack of randomization can lead to omitted variable bias.
                This is because the difference-in-differences estimate would
                attribute differences in growth trajectories to the treatment
                effect, when in fact they are driven by these omitted variables,
                which were left out of the regression and unobserved.
            \item \begin{enumerate}
                \item The regression output for household income per capita on
                    cash village status at baseline shows a coefficient of 
                    4.061 with a standard error of 11.437, resulting in a 
                    t-statistic of 0.355. This indicates that there is no 
                    statistically significant difference in household income 
                    per capita between ineligible households in cash villages
                    and control villages at baseline.

                    For the regression of having an elderly member on cash
                    village status at baseline, the coefficient is -0.026 with
                    a standard error of 0.024, yielding a t-statistic of -1.100.
                    This suggests that there is again no statistically 
                    significant difference in the proportion of households with 
                    elderly members between cash and control villages at 
                    baseline.

                    Taken together, these findings suggest that ineligible
                    households are relatively similar in cash villages versus
                    control villages along these two dimensions, which supports
                    the validity of the difference-in-differences approach.

                \item The difference-in-differences analysis estimating the
                    spillover impact of GD cash transfers on per capita income
                    yields a coefficient of 62.190 on the interaction term with 
                    a standard error of 33.512, resulting in a t-statistic of 
                    1.856. The estimate indicates that ineligible households 
                    in villages where the treatment was administered 
                    experienced roughly 62 higher per-capita incomes than those 
                    in control villages three years after the transfers.
                    This suggests that the spillover effect of cash 
                    transfers on ineligible households is positive but not 
                    statistically significant at the 95\% confidence level.
                    That is, we fail to reject the null hypothesis that the 
                    spillover effect is zero at the 5\% significance level.

                \item There are several reasons why cash transfers received by
                    other households in one's village might generate spillover
                    effects over time among non-recipient households. First,
                    cash transfers can increase overall economic activity in 
                    the village, leading to higher demand for goods and 
                    services that benefits non-recipient households. Second, 
                    the cash transfers might lead to increased saving and
                    investment by recipient households, which can create
                    employment opportunities for non-recipient households.
                    Given these mechanisms, the magnitude of the estimated 
                    effect in part (b) seems plausible, as the positive 
                    spillover effect aligns with the expected economic dynamics 
                    in the village following cash transfers to eligible 
                    households.
            \end{enumerate}
        \end{enumerate}
    \end{solution}

\end{homeworkProblem}