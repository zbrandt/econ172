\begin{homeworkProblem}{Weather and Witch Killing}

    This question builds on the econometric analysis in \textit{Miguel 2005} 
    about Tanzania poverty and witch-killing. You will carry out some 
    econometrics analysis related to that paper and also related to the 
    \textit{Miguel, Satyanath and Sergenti (2004)} article. You may write up 
    your answers using a word processor, include copies/screenshots of your 
    regression tables, and attach a copy of your R script at the end. 
    Alternatively, you may choose to produce an RMarkdown file that integrates 
    your code, your written responses, and tables displaying your regression 
    results into a single document (please ``knit to PDF'' and turn in the
    resulting PDF file; do not simply turn in your \texttt{.rmd} file). 
    
    In either case, your submission must include: Your entire R code/script; 
    Your written answers; Your regression output. Please merge all documents 
    into a single PDF before submitting.
    
    Please download ``pset3-2025-killing.csv'' from the bCourses page.
    
    Use the \texttt{read.csv} command to open it in R (or RStudio), either on 
    your local computer, or on UC Berkeley’s DataHub, found at 
    \href{https://r.datahub.berkeley.edu/}{https://r.datahub.berkeley.edu/}.
    
    This dataset is a partial extract of the data from \textit{Miguel (2005)}, 
    organized such that each observation (row) contains data for a particular 
    village (denoted by the variable vid) in a particular year (denoted year) 
    in Meatu district, Tanzania. In other words, this is panel data.

    Variables include:
    \begin{itemize}
        \item \texttt{witch\_murders}: number of witch murders in a given year
            village-year
        \item \texttt{oth\_murders}: number of non-witch murders
        \item \texttt{any\_rain}: indicator (1/0) for whether a drought or flood
            occurred
        \item \texttt{any\_disease}: indicator for whether a disease outbreak 
            (measles, cholera, etc.) occurred
        \item \texttt{famine}: indicator for whether there was an extreme food
            shortage
        \item \texttt{educat}: average years of schooling in the village
        \item \texttt{trad\_relig}: proportion of households practicing
            tradiational religions
    \end{itemize}

    \begin{itemize}
        \item[a)] Construct a new variable for the total number of murders in a 
            village-year (witch + non-witch murders).
        \item[b)] Create a table of summary statistics for all variables in the 
            dataset, including the mean, standard deviation, minimum, maximum, 
            and number of observations, using \texttt{stargazer}, 
            \texttt{summary} or \texttt{describe} commands in R. Discuss any 
            noteworthy patterns. Pay particular attention to the murder and 
            rainfall variables.
        \item[c)] Now consider the effect of extreme weather on murders in the 
            village.
            \begin{itemize}
                \item[(i)] Install ``miceadds'' and ``sandwich.'' Using the 
                    \texttt{lm.cluster} command, regress total murders (in a 
                    village in a particular year) on the indicator for whether 
                    a drought or flood occurred in that year. Make sure that 
                    error terms should be allowed to be correlated 
                    (``clustered'') across years for the same village (use 
                    \texttt{vid}). Simply use \texttt{summary} to report the 
                    results in this question. 
                    [Note: Results estimated by \texttt{lm.cluster} could not 
                    be exported directly with \texttt{stargazer} so we use 
                    \texttt{summary} for simplicity. In the section we will 
                    teach how to export clustered regression results in a 
                    neater way.]
                \item[(ii)] In a second regression, add average years of 
                    schooling and proportion of households practicing 
                    traditional religions as additional explanatory variables.
                \item[(iii)] Interpret both regressions carefully.
            \end{itemize}
        \item[d)] Finally, consider a possible instrumental variables (IV) 
            approach. Economic theory suggests that extreme economic 
            hardship—such as a famine—may be associated with more violence, 
            including murders. Famine may be caused by extreme rainfall (which 
            would be the instrumental variable).
            \begin{itemize}
                \item[(i)] Write out the first stage regression, the second 
                    stage regression, and the reduced form regression.
                \item[(ii)] Evaluate whether this is a valid IV approach by 
                    discussing the plausibility of the three key IV conditions: 
                    Relevance; Exclusion restriction; Exogeneity. What are some 
                    specific ways in which each of these assumptions might be 
                    appropriate or might fail in this context?
            \end{itemize}
    \end{itemize}

    \begin{solution}
        \begin{itemize}
            \item [a)] The new variable for total murders is created in R as 
                follows:
                \lstinputlisting[linerange={22-22}]{source/pset3.Rmd}

            \item [b)] Summary statistics for all variables in the dataset 
                are shown below. The mean number of witch murders (0.091) and 
                other murders (0.091) are identical, each occurring in roughly 
                9\% of village-years. Extreme rainfall events occur in 
                approximately 17\% of observations, while famines occur in 
                17.4\% of village-years. The strong correlation between these 
                suggests that weather shocks may be driving food insecurity. 
                On average, villagers have about 4 years of schooling, and 
                65\% of households practice traditional religions. The murder 
                variables are highly right-skewed, with most village-years 
                experiencing zero murders and maximum values of 3 for witch 
                murders and 5 for other murders.

                \lstinputlisting[linerange={32-32}]{source/pset3.Rmd}

                \begin{table}[!htbp] \centering \color{blue}    
                    \begin{tabular}{@{\extracolsep{5pt}}lccccc} 
                    \\[-1.8ex]\hline 
                    \hline \\[-1.8ex] 
                    Statistic & \multicolumn{1}{c}{N} & \multicolumn{1}{c}{Mean} & \multicolumn{1}{c}{St. Dev.} & \multicolumn{1}{c}{Min} & \multicolumn{1}{c}{Max} \\ 
                    \hline \\[-1.8ex] 
                    vid & 736 & 35.034 & 20.660 & 1 & 71 \\ 
                    year & 736 & 1,996.993 & 3.161 & 1,992 & 2,002 \\ 
                    witch\_murders & 736 & 0.091 & 0.323 & 0 & 3 \\ 
                    oth\_murders & 736 & 0.091 & 0.395 & 0 & 5 \\ 
                    any\_rain & 736 & 0.171 & 0.377 & 0 & 1 \\ 
                    any\_disease & 736 & 0.148 & 0.355 & 0 & 1 \\ 
                    famine & 736 & 0.174 & 0.379 & 0 & 1 \\ 
                    educat & 736 & 4.035 & 1.068 & 0.857 & 6.667 \\ 
                    trad\_relig & 736 & 0.654 & 0.206 & 0.000 & 1.000 \\ 
                    total\_murders & 736 & 0.182 & 0.516 & 0 & 5 \\ 
                    \hline \\[-1.8ex] 
                    \end{tabular} 
                \end{table}

            \item [c)]
                \begin{itemize}
                    \item [(i)] The first regression estimates the effect of 
                        extreme weather on total murders with clustered 
                        standard errors:
                        
                        \lstinputlisting[linerange={46-49}]{source/pset3.Rmd}
                        
                        The estimated regression equation is:
                        \[
                        \widehat{\text{total\_murders}}_{it} = 0.174 + 0.048 \cdot \text{any\_rain}_{it}
                        \]
                        \[
                        (0.022) \quad (0.046)
                        \]
                        
                        where standard errors (clustered by village) are in 
                        parentheses. The coefficient is not statistically 
                        significant ($p = 0.289$), and the $R^2 = 0.00125$ 
                        indicates that extreme rainfall explains very little 
                        variation in murders.

                    \item [(ii)] Adding education and traditional religion 
                        controls:
                        
                        \lstinputlisting[linerange={57-60}]{source/pset3.Rmd}
                        
                        The estimated regression equation is:
                        \[
                        \widehat{\text{total\_murders}}_{it} = 0.328 + 0.040 \cdot \text{any\_rain}_{it} - 0.038 \cdot \text{educat}_{it} + 0.001 \cdot \text{trad\_relig}_{it}
                        \]
                        \[
                        (0.146) \quad (0.043) \quad (0.026) \quad (0.104)
                        \]
                        
                        Again, none of the coefficients are statistically 
                        significant. The $R^2 = 0.00738$ remains very low.

                    \item [(iii)] Both regressions suggest no statistically 
                        significant relationship between extreme weather and 
                        murders. In the first regression, the point estimate 
                        suggests that extreme rainfall is associated with 
                        0.048 additional murders per village-year, but this 
                        effect is not statistically different from zero. 
                        Adding controls for education and traditional religion 
                        in the second regression slightly reduces the 
                        coefficient on rainfall to 0.040, and it remains 
                        insignificant. Education shows a negative relationship 
                        with murders (as expected), while traditional religion 
                        shows essentially no relationship. The very low $R^2$ 
                        values indicate that these variables explain almost 
                        none of the variation in murders, suggesting that 
                        other factors are more important determinants of 
                        violence in these villages.
                \end{itemize}

            \item [d)]
                \begin{itemize}
                    \item [(i)] In the proposed IV approach, extreme rainfall 
                        (\texttt{any\_rain}) serves as an instrument for 
                        famine. The three key regression equations are:
                        
                        \textbf{First stage regression} (effect of instrument 
                        on endogenous variable):
                        \[
                        \text{famine}_{it} = \pi_0 + \pi_1 \cdot \text{any\_rain}_{it} + v_{it}
                        \]
                        
                        \textbf{Second stage regression} (effect of 
                        instrumented variable on outcome):
                        \[
                        \text{total\_murders}_{it} = \beta_0 + \beta_1 \cdot \widehat{\text{famine}}_{it} + \epsilon_{it}
                        \]
                        
                        \textbf{Reduced form regression} (effect of instrument 
                        on outcome):
                        \[
                        \text{total\_murders}_{it} = \gamma_0 + \gamma_1 \cdot \text{any\_rain}_{it} + u_{it}
                        \]
                        
                        Note that $\gamma_1 = \pi_1 \times \beta_1$ in the IV 
                        framework. From our analysis, the first stage yields 
                        $\hat{\pi}_1 = 0.403$ ($p < 0.001$), showing a strong 
                        relationship between extreme rainfall and famine. The 
                        reduced form yields $\hat{\gamma}_1 = 0.054$ 
                        ($p = 0.413$), which is not statistically significant.

                    \item [(ii)] For this IV approach to be valid, three 
                        conditions must hold:
                        
                        \textbf{Relevance}: The instrument 
                        (\texttt{any\_rain}) must be sufficiently correlated 
                        with the endogenous variable (\texttt{famine}). This 
                        condition appears to be satisfied. The first stage 
                        regression shows a strong, statistically significant 
                        relationship ($\hat{\pi}_1 = 0.403$, $p < 0.001$) with 
                        $R^2 = 0.16$. The correlation between extreme rainfall 
                        and famine is 0.40, and the $t$-statistic of 6.64 far 
                        exceeds conventional thresholds for weak instruments. 
                        Economically, this makes sense: droughts and floods 
                        destroy crops and livestock, directly causing food 
                        shortages.
                        
                        \textbf{Exclusion restriction}: The instrument should 
                        affect the outcome (murders) \textit{only} through the 
                        endogenous variable (famine), not through any other 
                        channel. This is the most questionable assumption in 
                        this context. Extreme weather could affect violence 
                        through mechanisms other than famine: (1) Rainfall 
                        extremes might directly impact mental health or stress 
                        levels, leading to conflict; (2) Floods or droughts 
                        could disrupt economic activities beyond agriculture, 
                        such as trade or labor markets; (3) Weather shocks 
                        might trigger migration or displacement, altering 
                        village social dynamics; (4) Disease outbreaks 
                        (included in the data) often follow floods, providing 
                        an alternative channel for scapegoating and violence. 
                        The fact that the reduced form coefficient is positive 
                        but insignificant (0.054, $p = 0.413$) suggests either 
                        a weak overall effect or that multiple channels may be 
                        offsetting each other.
                        
                        \textbf{Exogeneity}: The instrument must be uncorrelated 
                        with unobserved determinants of the outcome. This 
                        condition is plausible since rainfall is largely 
                        determined by climatic factors beyond human control. 
                        However, it could be violated if: (1) Villages with 
                        certain unobserved characteristics (e.g., proximity to 
                        water bodies, elevation) are both more prone to extreme 
                        rainfall and have different baseline violence rates; 
                        (2) There are omitted time-varying factors correlated 
                        with both weather patterns and violence (e.g., regional 
                        conflicts, political instability). The panel structure 
                        of the data helps address time-invariant village 
                        characteristics, but time-varying confounders remain a 
                        concern.
                        
                        Overall, while the relevance condition is well-satisfied, 
                        the exclusion restriction is questionable due to 
                        multiple plausible channels through which weather could 
                        affect violence. The exogeneity assumption is 
                        reasonably plausible given the random nature of weather 
                        shocks, though not guaranteed. These concerns suggest 
                        caution in interpreting any IV estimates as causal 
                        effects.
                \end{itemize}
        \end{itemize}
    \end{solution}

\end{homeworkProblem}