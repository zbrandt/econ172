\begin{homeworkProblem}{Weather and Witch Killing}

    This question builds on the econometric analysis in \textit{Miguel 2005} 
    about Tanzania poverty and witch-killing. You will carry out some 
    econometrics analysis related to that paper and also related to the 
    \textit{Miguel, Satyanath and Sergenti (2004)} article. You may write up 
    your answers using a word processor, include copies/screenshots of your 
    regression tables, and attach a copy of your R script at the end. 
    Alternatively, you may choose to produce an RMarkdown file that integrates 
    your code, your written responses, and tables displaying your regression 
    results into a single document (please ``knit to PDF'' and turn in the
    resulting PDF file; do not simply turn in your \texttt{.rmd} file). 
    
    In either case, your submission must include: Your entire R code/script; 
    Your written answers; Your regression output. Please merge all documents 
    into a single PDF before submitting.
    
    Please download ``pset3-2025-killing.csv'' from the bCourses page.
    
    Use the \texttt{read.csv} command to open it in R (or RStudio), either on 
    your local computer, or on UC Berkeley’s DataHub, found at 
    \href{https://r.datahub.berkeley.edu/}{https://r.datahub.berkeley.edu/}.
    
    This dataset is a partial extract of the data from \textit{Miguel (2005)}, 
    organized such that each observation (row) contains data for a particular 
    village (denoted by the variable vid) in a particular year (denoted year) 
    in Meatu district, Tanzania. In other words, this is panel data.

    Variables include:
    \begin{itemize}
        \item \texttt{witch\_murders}: number of witch murders in a given year
            village-year
        \item \texttt{oth\_murders}: number of non-witch murders
        \item \texttt{any\_rain}: indicator (1/0) for whether a drought or flood
            occurred
        \item \texttt{any\_disease}: indicator for whether a disease outbreak 
            (measles, cholera, etc.) occurred
        \item \texttt{famine}: indicator for whether there was an extreme food
            shortage
        \item \texttt{educat}: average years of schooling in the village
        \item \texttt{trad\_relig}: proportion of households practicing
            tradiational religions
    \end{itemize}

    \begin{itemize}
        \item[a)] Construct a new variable for the total number of murders in a 
            village-year (witch + non-witch murders).
        \item[b)] Create a table of summary statistics for all variables in the 
            dataset, including the mean, standard deviation, minimum, maximum, 
            and number of observations, using \texttt{stargazer}, 
            \texttt{summary} or \texttt{describe} commands in R. Discuss any 
            noteworthy patterns. Pay particular attention to the murder and 
            rainfall variables.
        \item[c)] Now consider the effect of extreme weather on murders in the 
            village.
            \begin{itemize}
                \item[(i)] Install ``miceadds'' and ``sandwich.'' Using the 
                    \texttt{lm.cluster} command, regress total murders (in a 
                    village in a particular year) on the indicator for whether 
                    a drought or flood occurred in that year. Make sure that 
                    error terms should be allowed to be correlated 
                    (``clustered'') across years for the same village (use 
                    \texttt{vid}). Simply use \texttt{summary} to report the 
                    results in this question. 
                    [Note: Results estimated by \texttt{lm.cluster} could not 
                    be exported directly with \texttt{stargazer} so we use 
                    \texttt{summary} for simplicity. In the section we will 
                    teach how to export clustered regression results in a 
                    neater way.]
                \item[(ii)] In a second regression, add average years of 
                    schooling and proportion of households practicing 
                    traditional religions as additional explanatory variables.
                \item[(iii)] Interpret both regressions carefully.
            \end{itemize}
        \item[d)] Finally, consider a possible instrumental variables (IV) 
            approach. Economic theory suggests that extreme economic 
            hardship—such as a famine—may be associated with more violence, 
            including murders. Famine may be caused by extreme rainfall (which 
            would be the instrumental variable).
            \begin{itemize}
                \item[(i)] Write out the first stage regression, the second 
                    stage regression, and the reduced form regression.
                \item[(ii)] Evaluate whether this is a valid IV approach by 
                    discussing the plausibility of the three key IV conditions: 
                    Relevance; Exclusion restriction; Exogeneity. What are some 
                    specific ways in which each of these assumptions might be 
                    appropriate or might fail in this context?
            \end{itemize}
    \end{itemize}

    \begin{solution}
        \begin{itemize}
            \item [a)] The new variable for total murders is created in R as 
                follows:
                \lstinputlisting[linerange={22-22}]{source/pset3.Rmd}

            \item [b)] Summary statistics for all variables in the dataset 
                are shown below. 

                \begin{table}[!htbp] \centering \color{blue}    
                    \begin{tabular}{@{\extracolsep{5pt}}lccccc} 
                    \\[-1.8ex]\hline 
                    \hline \\[-1.8ex] 
                    Statistic & \multicolumn{1}{c}{N} & 
                    \multicolumn{1}{c}{Mean} & \multicolumn{1}{c}{St. Dev.} & 
                    \multicolumn{1}{c}{Min} & \multicolumn{1}{c}{Max} \\ 
                    \hline \\[-1.8ex] 
                    vid & 736 & 35.034 & 20.660 & 1 & 71 \\ 
                    year & 736 & 1,996.993 & 3.161 & 1,992 & 2,002 \\ 
                    witch\_murders & 736 & 0.091 & 0.323 & 0 & 3 \\ 
                    oth\_murders & 736 & 0.091 & 0.395 & 0 & 5 \\ 
                    any\_rain & 736 & 0.171 & 0.377 & 0 & 1 \\ 
                    any\_disease & 736 & 0.148 & 0.355 & 0 & 1 \\ 
                    famine & 736 & 0.174 & 0.379 & 0 & 1 \\ 
                    educat & 736 & 4.035 & 1.068 & 0.857 & 6.667 \\ 
                    trad\_relig & 736 & 0.654 & 0.206 & 0.000 & 1.000 \\ 
                    total\_murders & 736 & 0.182 & 0.516 & 0 & 5 \\ 
                    \hline \\[-1.8ex] 
                    \end{tabular} 
                \end{table}

                The mean number of witch murders (0.091) and 
                other murders (0.091) are identical. Droughts or floods 
                occur in approximately 17\% of observations, while extreme food
                shortages occur in 17.4\% of village-years. The strong 
                correlation between these suggests that weather shocks may be 
                driving food insecurity. On average, villagers have about 4 
                years of schooling, and 65\% of households practice traditional 
                religions. The murder variables are right-skewed, with most 
                village-years experiencing zero murders and maximum values of 3 
                for witch murders and 5 for other murders.

            \item [c)]
                \begin{itemize}
                    \item [(i)] The first regression estimates the effect of 
                        extreme weather on total murders with clustered 
                        standard errors:
                        \[
                            \text{total\_murders}_{it} = \beta_0 + \beta_1 
                            \cdot \text{any\_rain}_{it} + \epsilon_{it}  
                        \]
                        The estimated regression equation is
                        \[
                            \widehat{\text{total\_murders}}_{it} = 0.174 + 
                            0.048 \cdot \text{any\_rain}_{it}
                        \]
                        
                        where standard errors (clustered by village) are 0.022
                        for the constant and 0.046 for any\_rain. The 
                        coefficient is not statistically significant ($p = 
                        0.289$), and the $R^2 = 0.00125$ indicates that extreme 
                        rainfall explains very little variation in murders.

                    \item [(ii)] Adding education and traditional religion 
                        controls:
                        \[
                            \text{total\_murders}_{it} = \beta_0 + \beta_1 
                            \cdot \text{any\_rain}_{it} + \beta_2 \cdot 
                            \text{educat}_{it} + \beta_3 \cdot 
                            \text{trad\_relig}_{it} + \epsilon_{it}
                        \]
                        
                        The estimated regression equation is:
                        \[
                            \widehat{\text{total\_murders}}_{it} = 0.328 + 
                            0.040 \cdot \text{any\_rain}_{it} - 0.038 \cdot 
                            \text{educat}_{it} + 0.001 \cdot 
                            \text{trad\_relig}_{it}
                        \]
                        
                        where standard errors (clustered by village) are 0.146 
                        for the constant, 0.043 for any\_rain, 0.026 for 
                        educat, and 0.104 for trad\_relig. Again, none of the 
                        coefficients are statistically significant, with all
                        having $p$-values greater than 0.05. The $R^2 = 
                        0.00738$ remains low.

                    \item [(iii)] Both regressions suggest no statistically 
                        significant relationship between extreme weather and 
                        murders. In the first regression, the point estimate 
                        suggests that extreme rainfall is associated with 
                        0.048 additional murders per village-year, but this 
                        effect is not statistically different from zero. 
                        Adding controls for education and traditional religion 
                        in the second regression slightly reduces the 
                        coefficient on rainfall to 0.040, and it remains 
                        insignificant. Education shows a negative relationship 
                        with murders (as expected), while traditional religion 
                        shows essentially no relationship. The very low $R^2$ 
                        values indicate that these variables explain almost 
                        none of the variation in murders, suggesting that 
                        other factors are more important determinants of 
                        violence in these villages.
                \end{itemize}

            \item [d)]
                \begin{itemize}
                    \item [(i)] In the proposed IV approach, extreme rainfall 
                        (\texttt{any\_rain}) serves as an instrument for 
                        famine. The first-stage regression is
                        \[
                            \text{famine}_{it} = \pi_0 + \pi_1 \cdot 
                            \text{any\_rain}_{it} + v_{it}
                        \]
                        which estimates the effect of extreme rainfall on the 
                        likelihood of famine. The second-stage regression is
                        \[
                            \text{total\_murders}_{it} = \beta_0 + \beta_1 
                            \cdot \widehat{\text{famine}}_{it} + \epsilon_{it}
                        \]
                        which estimates the effect of famine (predicted from
                        the first stage) on total murders. The reduced form
                        regression is
                        \[
                            \text{total\_murders}_{it} = \gamma_0 + \gamma_1 
                            \cdot \text{any\_rain}_{it} + u_{it}
                        \]
                        which directly estimates the effect of extreme rainfall
                        on total murders.

                    \item [(ii)] For this IV approach to be valid, three 
                        conditions must hold:
                        
                        \textbf{Relevance}: The instrument needs to be related
                        strongly enough to the endogenous explanatory variable 
                        of interest. In this case, extreme rainfall should have 
                        a significant effect on the likelihood of famine. This 
                        seems appropriate, as extreme weather events can 
                        disrupt agricultural production and food availability.
                        However, this might fail if there are other factors 
                        (e.g., government aid, market access) that mitigate the
                        impact of rainfall, avoiding famine.
                
                        \textbf{Exclusion Restriction}: The instrument must not
                        have a direct effect on the dependent variable 
                        except through its relationship with the endogenous 
                        explanatory variable. In this case, extreme rainfall 
                        should affect murders only through its impact on famine. 
                        This assumption could be appropriate if we believe that
                        rainfall does not directly influence violence, but only
                        through its impact on famine. However, this might fail 
                        if extreme weather also causes other stressors (e.g.,
                        displacement, resource conflicts) that directly increase
                        violence, violating the exclusion restriction.

                        \textbf{Exogeneity}: The instrument must be 
                        uncorrelated with any unobserved variables (in the 
                        error term) that also affect the dependent variable. In 
                        this case, extreme rainfall should not be correlated with
                        other unobserved factors that influence murders. This
                        assumption might be appropriate if weather patterns are
                        random and not influenced by local social or economic
                        conditions. However, this might fail if certain villages
                        are more prone to both extreme weather and violence due
                        to unobserved characteristics (e.g., geographic features,
                        historical conflicts), leading to correlation between
                        the instrument and the error term.
                        
                \end{itemize}
        \end{itemize}
    \end{solution}

\end{homeworkProblem}