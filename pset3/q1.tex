\begin{homeworkProblem}{Weather and Witch Killing}

    This question builds on the econometric analysis in \textit{Miguel 2005} 
    about Tanzania poverty and witch-killing. You will carry out some 
    econometrics analysis related to that paper and also related to the 
    \textit{Miguel, Satyanath and Sergenti (2004)} article. You may write up 
    your answers using a word processor, include copies/screenshots of your 
    regression tables, and attach a copy of your R script at the end. 
    Alternatively, you may choose to produce an RMarkdown file that integrates 
    your code, your written responses, and tables displaying your regression 
    results into a single document (please ``knit to PDF'' and turn in the
    resulting PDF file; do not simply turn in your \texttt{.rmd} file). 
    
    In either case, your submission must include: Your entire R code/script; 
    Your written answers; Your regression output. Please merge all documents 
    into a single PDF before submitting.
    
    Please download ``pset3-2025-killing.csv'' from the bCourses page.
    
    Use the \texttt{read.csv} command to open it in R (or RStudio), either on 
    your local computer, or on UC Berkeley’s DataHub, found at 
    \href{https://r.datahub.berkeley.edu/}{https://r.datahub.berkeley.edu/}.
    
    This dataset is a partial extract of the data from \textit{Miguel (2005)}, 
    organized such that each observation (row) contains data for a particular 
    village (denoted by the variable vid) in a particular year (denoted year) 
    in Meatu district, Tanzania. In other words, this is panel data.

    Variables include:
    \begin{itemize}
        \item \texttt{witch\_murders}: number of witch murders in a given year
            village-year
        \item \texttt{oth\_murders}: number of non-witch murders
        \item \texttt{any\_rain}: indicator (1/0) for whether a drought or flood
            occurred
        \item \texttt{any\_disease}: indicator for whether a disease outbreak 
            (measles, cholera, etc.) occurred
        \item \texttt{famine}: indicator for whether there was an extreme food
            shortage
        \item \texttt{educat}: average years of schooling in the village
        \item \texttt{trad\_relig}: proportion of households practicing
            tradiational religions
    \end{itemize}

    \begin{itemize}
        \item[a)] Construct a new variable for the total number of murders in a 
            village-year (witch + non-witch murders).
        \item[b)] Create a table of summary statistics for all variables in the 
            dataset, including the mean, standard deviation, minimum, maximum, 
            and number of observations, using \texttt{stargazer}, 
            \texttt{summary} or \texttt{describe} commands in R. Discuss any 
            noteworthy patterns. Pay particular attention to the murder and 
            rainfall variables.
        \item[c)] Now consider the effect of extreme weather on murders in the 
            village.
            \begin{itemize}
                \item[(i)] Install ``miceadds'' and ``sandwich.'' Using the 
                    \texttt{lm.cluster} command, regress total murders (in a 
                    village in a particular year) on the indicator for whether 
                    a drought or flood occurred in that year. Make sure that 
                    error terms should be allowed to be correlated 
                    (``clustered'') across years for the same village (use 
                    \texttt{vid}). Simply use \texttt{summary} to report the 
                    results in this question. 
                    [Note: Results estimated by \texttt{lm.cluster} could not 
                    be exported directly with \texttt{stargazer} so we use 
                    \texttt{summary} for simplicity. In the section we will 
                    teach how to export clustered regression results in a 
                    neater way.]
                \item[(ii)] In a second regression, add average years of 
                    schooling and proportion of households practicing 
                    traditional religions as additional explanatory variables.
                \item[(iii)] Interpret both regressions carefully.
            \end{itemize}
        \item[d)] Finally, consider a possible instrumental variables (IV) 
            approach. Economic theory suggests that extreme economic 
            hardship—such as a famine—may be associated with more violence, 
            including murders. Famine may be caused by extreme rainfall (which 
            would be the instrumental variable).
            \begin{itemize}
                \item[(i)] Write out the first stage regression, the second 
                    stage regression, and the reduced form regression.
                \item[(ii)] Evaluate whether this is a valid IV approach by 
                    discussing the plausibility of the three key IV conditions: 
                    Relevance; Exclusion restriction; Exogeneity. What are some 
                    specific ways in which each of these assumptions might be 
                    appropriate or might fail in this context?
            \end{itemize}
    \end{itemize}

\end{homeworkProblem}