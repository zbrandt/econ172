\begin{homeworkProblem}{The Primary School Deworming Project (PSDP)}
    
    For this assignment, we will analyze the dataset used in the paper titled 
    ``Worms at Work: Long-Run Impacts of a Child Health Investment.''

    Please download the dataset ``pset3-2025-deworming.csv'' from bCourses.

    In this question, you will estimate the treatment effects of deworming for 
    the following dependent variables:
    \begin{enumerate}
        \item Total years enrolled in school, 1998--2007 
            (\texttt{totyrs\_enrolled})
        \item Indicator for passed secondary school entrance exam 
            (\texttt{passed\_primary\_exam})
        \item Number of meals eaten yesterday (\texttt{num\_meals\_yesterday})
        \item Total hours worked in wages/self-employment/agriculture, last 7 
            days (\texttt{total\_hours})
        \item Wages for total cash salary/food in kind, last month 
            (\texttt{ln\_emp\_salary\_total})
    \end{enumerate}

    The treatment variable varies at the school level and is called 
    \texttt{treatment}.

    The authors also include the following control variables:
\begin{verbatim}
saturation_dm+demeaned_popT_6k+zoneidI2+zoneidI3+zoneidI4+zoneidI5+zoneidI6+z
oneidI7+zoneidI8+pup_pop+month_interviewI2+month_interviewI3+month_inte
rviewI4+month_interviewI5+month_interviewI6+month_interviewI7+month_interview
I8+month_interviewI9+month_interviewI10+month_interviewI11+month_interviewI12
+cost_sharing+std98_base_I2+std98_base_I3+std98_base_I4+std98_base_I5+std98_b
ase_I6+female_baseline+avgtest96
\end{verbatim}

    Use sampling weights \texttt{weight} in your regressions.

    Cluster standard errors at the school level using the variable 
    \texttt{psdpsch98}.

    \begin{itemize}
        \item[a)] In this question we use linear regression to estimate the 
            effect of the deworming treatment on the five dependent variables 
            mentioned above.
            \begin{itemize}
                \item [(i)] Use R to estimate regressions in the format of the 
                    following. Simply use \texttt{summary} to report the 
                    results in this question. \textbf{To receive full credits, 
                    please highlight the names of dependent variables and 
                    estimated coefficients of \texttt{treatment} with red 
                    rectangles.} You could do this by annotating the pdf 
                    document compiled from .rmd or exported from Microsoft 
                    Word.  
\begin{verbatim}
name_dep_var = treatment + name_control_vars, with "weight"
as the sampling weight and "psdpsch98" as the cluster ID
\end{verbatim}

                \item [(ii)] Interpret the treatment effect coefficient for 
                    the regressions on total years enrolled and passing the 
                    secondary school exam.
            \end{itemize}

        \item[b)] Deworming benefits might be stronger for certain groups --- 
            for instance, girls (perhaps because they were more likely to be 
            infected) or children with lower BMI at baseline (because they were 
            less healthy initially).
            \begin{itemize}
                \item [(i)] Please estimate whether the deworming treatment had 
                    a differential impact on \texttt{totyrs\_enrolled}, 
                    \texttt{passed\_primary\_exam}, and \texttt{total\_hours} 
                    by gender (\texttt{female\_baseline}) and then by BMI 
                    (\texttt{BMI}). \textbf{To receive full credits, please 
                    highlight the names of dependent variables and estimated 
                    coefficients of interactive terms with red rectangles.}
                \item[(ii)] Indicate which regressions have a significant 
                    interaction term at the 10\% level. Interpret the 
                    coefficient on the interaction term for these regressions.
            \end{itemize}
    \end{itemize}

    \begin{solution}
        
        \begin{itemize}
            \item[a)]
            \begin{itemize}
                \item[(i)] Five weighted regressions with clustered standard 
                    errors were estimated. Each regression takes the form:
                    \[
                    Y_i = \beta_0 + \beta_1 \cdot \text{treatment}_i + 
                    \mathbf{X}_i'\boldsymbol{\gamma} + \epsilon_i
                    \]
                    where $Y_i$ is the dependent variable, $\text{treatment}_i$ 
                    is the deworming treatment indicator, and $\mathbf{X}_i$ 
                    includes all control variables. Standard errors are 
                    clustered at the school level (psdpsch98), and 
                    observations are weighted using the sampling weight.
                    
                    \textbf{Regression Results:}
                    
                    \textbf{(1) Total years enrolled in school 
                    (totyrs\_enrolled):}
                    \begin{itemize}
                        \item Treatment coefficient: $\hat{\beta}_1 = 0.293$ 
                            (SE = 0.145)
                        \item $t$-value = 2.026, $p$-value = 0.043
                        \item $R^2 = 0.293$
                    \end{itemize}
                    
                    \textbf{(2) Passed primary school entrance exam 
                    (passed\_primary\_exam):}
                    \begin{itemize}
                        \item Treatment coefficient: $\hat{\beta}_1 = 0.051$ 
                            (SE = 0.031)
                        \item $t$-value = 1.641, $p$-value = 0.101
                        \item $R^2 = 0.070$
                    \end{itemize}
                    
                    \textbf{(3) Number of meals eaten yesterday 
                    (num\_meals\_yesterday):}
                    \begin{itemize}
                        \item Treatment coefficient: $\hat{\beta}_1 = 0.095$ 
                            (SE = 0.029)
                        \item $t$-value = 3.337, $p$-value = 0.001
                        \item $R^2 = 0.034$
                    \end{itemize}
                    
                    \textbf{(4) Total hours worked (total\_hours):}
                    \begin{itemize}
                        \item Treatment coefficient: $\hat{\beta}_1 = 1.599$ 
                            (SE = 1.036)
                        \item $t$-value = 1.543, $p$-value = 0.123
                        \item $R^2 = 0.059$
                    \end{itemize}
                    
                    \textbf{(5) Log employment salary 
                    (ln\_emp\_salary\_total):}
                    \begin{itemize}
                        \item Treatment coefficient: $\hat{\beta}_1 = 0.265$ 
                            (SE = 0.085)
                        \item $t$-value = 3.105, $p$-value = 0.002
                        \item $R^2 = 0.189$
                    \end{itemize}
                    
                    The full regression output with all control variables is 
                    included in the attached RMarkdown output (pset3.pdf in 
                    the source folder). The dependent variables and treatment 
                    coefficients are highlighted with red rectangles.
                
                \item[(ii)] \textbf{Interpretation of treatment effects:}
                
                \textbf{Total years enrolled in school (totyrs\_enrolled):} 
                The treatment coefficient is 0.293 with a p-value of 0.043 
                (statistically significant at the 5\% level). This means that 
                students in schools that received the deworming treatment 
                were enrolled in school for approximately 0.29 additional 
                years (about 3.5 months) compared to students in control 
                schools, holding all other factors constant. This represents 
                a meaningful increase in educational attainment attributable 
                to the deworming program.
                
                \textbf{Passed primary school entrance exam 
                (passed\_primary\_exam):} The treatment coefficient is 0.051 
                with a p-value of 0.101 (not statistically significant at the 
                10\% level). While the point estimate suggests that treated 
                students were 5.1 percentage points more likely to pass the 
                primary school entrance exam, this effect is not statistically 
                distinguishable from zero at conventional significance levels. 
                We cannot conclude with confidence that the deworming 
                treatment had an effect on exam passage rates.
            \end{itemize}
            
            \item[b)]
            \begin{itemize}
                \item[(i)] Six interaction regressions were estimated to test 
                    for differential treatment effects by gender and BMI. 
                    Each regression takes the form:
                    \[
                    Y_i = \beta_0 + \beta_1 \cdot \text{treatment}_i + 
                    \beta_2 \cdot Z_i + \beta_3 \cdot 
                    (\text{treatment}_i \times Z_i) + \mathbf{X}_i'
                    \boldsymbol{\gamma} + \epsilon_i
                    \]
                    where $Z_i$ is either female\_baseline or BMI, and 
                    $\beta_3$ is the interaction coefficient of interest.
                    
                    \textbf{Gender Interaction Results 
                    (treatment $\times$ female\_baseline):}
                    
                    \textbf{(1) Total years enrolled (totyrs\_enrolled):}
                    \begin{itemize}
                        \item Main treatment effect: $\hat{\beta}_1 = 0.324$ 
                            (SE = 0.181), $p = 0.073$
                        \item Interaction term: $\hat{\beta}_3 = -0.064$ 
                            (SE = 0.218), $p = 0.768$
                        \item $R^2 = 0.293$
                    \end{itemize}
                    
                    \textbf{(2) Passed primary exam (passed\_primary\_exam):}
                    \begin{itemize}
                        \item Main treatment effect: $\hat{\beta}_1 = 0.050$ 
                            (SE = 0.033), $p = 0.133$
                        \item Interaction term: $\hat{\beta}_3 = 0.001$ 
                            (SE = 0.040), $p = 0.976$
                        \item $R^2 = 0.070$
                    \end{itemize}
                    
                    \textbf{(3) Total hours worked (total\_hours):}
                    \begin{itemize}
                        \item Main treatment effect: $\hat{\beta}_1 = 3.506$ 
                            (SE = 1.480), $p = 0.018$
                        \item Interaction term: $\hat{\beta}_3 = -3.980$ 
                            (SE = 2.007), $p = 0.047$
                        \item $R^2 = 0.061$
                    \end{itemize}
                    
                    \textbf{BMI Interaction Results (treatment $\times$ BMI):}
                    
                    \textbf{(4) Total years enrolled (totyrs\_enrolled):}
                    \begin{itemize}
                        \item Main treatment effect: $\hat{\beta}_1 = 0.444$ 
                            (SE = 0.174), $p = 0.011$
                        \item Interaction term: $\hat{\beta}_3 = -0.006$ 
                            (SE = 0.005), $p = 0.197$
                        \item $R^2 = 0.294$
                    \end{itemize}
                    
                    \textbf{(5) Passed primary exam (passed\_primary\_exam):}
                    \begin{itemize}
                        \item Main treatment effect: $\hat{\beta}_1 = 0.065$ 
                            (SE = 0.034), $p = 0.057$
                        \item Interaction term: $\hat{\beta}_3 = -0.001$ 
                            (SE = 0.001), $p = 0.316$
                        \item $R^2 = 0.071$
                    \end{itemize}
                    
                    \textbf{(6) Total hours worked (total\_hours):}
                    \begin{itemize}
                        \item Main treatment effect: $\hat{\beta}_1 = -0.088$ 
                            (SE = 1.322), $p = 0.947$
                        \item Interaction term: $\hat{\beta}_3 = 0.075$ 
                            (SE = 0.039), $p = 0.054$
                        \item $R^2 = 0.060$
                    \end{itemize}
                    
                    The full regression output with all control variables is 
                    included in the attached RMarkdown output. The interaction 
                    coefficients are highlighted with red rectangles.
                
                \item[(ii)] \textbf{Significant interaction terms at the 10\% 
                    level:}
                
                Two regressions have significant or marginally significant 
                interaction terms:
                
                \textbf{1. Total hours worked × Gender 
                (treatment:female\_baseline):} The interaction coefficient is 
                $-3.98$ with a $p$-value of 0.047 (significant at the 5\% 
                level). 
                
                \textbf{Interpretation:} For males (female\_baseline = 0), 
                the deworming treatment increased hours worked by 3.51 hours 
                per week ($p = 0.018$, statistically significant). However, 
                for females, the differential effect is $-3.98$ hours, making 
                the total effect for females approximately $3.51 - 3.98 = 
                -0.47$ hours (essentially zero). This indicates that the 
                positive labor supply effect of deworming was concentrated 
                among males, while females experienced no significant change 
                in hours worked. This differential effect could reflect 
                gender differences in labor market participation, household 
                responsibilities, or the types of work opportunities 
                available to men versus women in this context.
                
                \textbf{2. Total hours worked × BMI (treatment:BMI):} The 
                interaction coefficient is $0.075$ with a $p$-value of 0.054 
                (marginally significant at the 10\% level).
                
                \textbf{Interpretation:} This positive interaction term 
                suggests that children with higher BMI at baseline 
                experienced larger increases in hours worked from the 
                deworming treatment. Specifically, for each one-unit increase 
                in baseline BMI, the treatment effect on hours worked 
                increased by approximately 0.075 hours per week. At the mean 
                BMI level, the treatment effect would be close to zero 
                (given the main effect of $-0.088$), but for children with 
                BMI one standard deviation above the mean, the positive 
                interaction effect would result in a positive net treatment 
                effect. This finding is somewhat counterintuitive since we 
                might expect children with lower BMI (who were less healthy) 
                to benefit more. However, it could suggest that healthier 
                children were better positioned to translate improved health 
                from deworming into increased labor supply, perhaps because 
                they had more energy reserves or faced fewer other health 
                constraints that would limit their ability to work more hours.
                
                \textbf{Other interactions:} None of the other interaction 
                terms were statistically significant at the 10\% level. The 
                interactions for total years enrolled and passed primary exam 
                with both gender and BMI showed no evidence of differential 
                treatment effects, indicating that the educational benefits 
                of deworming were similar across these demographic groups.
            \end{itemize}
        \end{itemize}

    \end{solution}

\end{homeworkProblem}