\begin{homeworkProblem}{The Primary School Deworming Project (PSDP)}
    
    For this assignment, we will analyze the dataset used in the paper titled 
    ``Worms at Work: Long-Run Impacts of a Child Health Investment.''

    Please download the dataset ``pset3-2025-deworming.csv'' from bCourses.

    In this question, you will estimate the treatment effects of deworming for 
    the following dependent variables:

    \begin{enumerate}
        \item Total years enrolled in school, 1998--2007 (\texttt{totyrs\_enrolled})
        \item Indicator for passed secondary school entrance exam 
            (\texttt{passed\_primary\_exam})
        \item Number of meals eaten yesterday (\texttt{num\_meals\_yesterday})
        \item Total hours worked in wages/self-employment/agriculture, last 7 days 
            (\texttt{total\_hours})
        \item Wages for total cash salary/food in kind, last month 
            (\texttt{ln\_emp\_salary\_total})
    \end{enumerate}

    The treatment variable varies at the school level and is called 
    \texttt{treatment}.

    The authors also include the following control variables:
    \begin{verbatim}
saturation_dm+demeaned_popT_6k+zoneidI2+zoneidI3+zoneidI4+zoneidI5+zoneidI6+z
oneidI7+zoneidI8+pup_pop+month_interviewI2+month_interviewI3+month_inte
rviewI4+month_interviewI5+month_interviewI6+month_interviewI7+month_interview
I8+month_interviewI9+month_interviewI10+month_interviewI11+month_interviewI12
+cost_sharing+std98_base_I2+std98_base_I3+std98_base_I4+std98_base_I5+std98_b
ase_I6+female_baseline+avgtest96
    \end{verbatim}

    Use sampling weights \texttt{weight} in your regressions.

    Cluster standard errors at the school level using the variable 
    \texttt{psdpsch98}.

    \textbf{a)} In this question we use linear regression to estimate the effect 
    of the deworming treatment on the five dependent variables mentioned above.

    \begin{enumerate}
        \item[(i)] Use R to estimate regressions in the format of the following. 
            Simply use \texttt{summary} to report the results in this question. 
            \textbf{To receive full credits, please highlight the names of 
            dependent variables and estimated coefficients of \texttt{treatment} 
            with red rectangles.} You could do this by annotating the pdf 
            document compiled from .rmd or exported from Microsoft Word.

            \begin{verbatim}
name_dep_var = treatment + name_control_vars, with "weight" as the sampling 
weight and "psdpsch98" as the cluster ID
            \end{verbatim}

        \item[(ii)] Interpret the treatment effect coefficient for the 
            regressions on total years enrolled and passing the secondary 
            school exam.
    \end{enumerate}

    \textbf{b)} Deworming benefits might be stronger for certain groups --- for 
    instance, girls (perhaps because they were more likely to be infected) or 
    children with lower BMI at baseline (because they were less healthy 
    initially).

    \begin{enumerate}
        \item[(i)] Please estimate whether the deworming treatment had a 
            differential impact on \texttt{totyrs\_enrolled}, 
            \texttt{passed\_primary\_exam}, and \texttt{total\_hours} by gender 
            (\texttt{female\_baseline}) and then by BMI (\texttt{BMI}). 
            \textbf{To receive full credits, please highlight the names of 
            dependent variables and estimated coefficients of interactive terms 
            with red rectangles.}
        \item[(ii)] Indicate which regressions have a significant interaction 
            term at the 10\% level. Interpret the coefficient on the interaction 
            term for these regressions.
    \end{enumerate}

    \begin{solution}
        
        \begin{enumerate}
            \item \begin{enumerate}
                \item[(i)] Below are the regression results for the five 
                    dependent variables using the \texttt{lm.cluster} command.

                    \textit{[See R output in compiled document]}

                    Summary of treatment effect coefficients:
                    \begin{itemize}
                        \item \textbf{Total years enrolled:} $\beta = 0.293$ 
                            (SE = 0.145, p = 0.043**)
                        \item \textbf{Passed primary exam:} $\beta = 0.051$ 
                            (SE = 0.031, p = 0.101)
                        \item \textbf{Number of meals yesterday:} $\beta = 0.095$ 
                            (SE = 0.029, p = 0.001***)
                        \item \textbf{Total hours worked:} $\beta = 1.599$ 
                            (SE = 1.036, p = 0.123)
                        \item \textbf{Log employment salary:} $\beta = 0.265$ 
                            (SE = 0.085, p = 0.002***)
                    \end{itemize}
                    
                    Note: * p $<$ 0.10, ** p $<$ 0.05, *** p $<$ 0.01

                \item[(ii)] The treatment effect coefficient for total years 
                    enrolled is 0.293 (SE = 0.145, p = 0.043), indicating that 
                    students in schools assigned to the deworming treatment 
                    were enrolled in school for approximately 0.29 additional 
                    years (about 3.5 months) compared to students in control 
                    schools, holding all control variables constant. This 
                    effect is statistically significant at the 5\% level.

                    The treatment effect coefficient for passing the secondary 
                    school entrance exam is 0.051 (SE = 0.031, p = 0.101), 
                    indicating that students in schools assigned to the 
                    deworming treatment were approximately 5.1 percentage 
                    points more likely to pass the secondary school entrance 
                    exam compared to students in control schools, holding all 
                    control variables constant. However, this effect is not 
                    statistically significant at the 10\% level (p-value of 
                    0.101 is just above the 0.10 threshold).

            \end{enumerate}

            \item \begin{enumerate}
                \item[(i)] Below are the regression results examining 
                    differential treatment effects by gender and BMI.

                    \textit{[See R output in compiled document]}

                    Summary of interaction term coefficients:

                    \textbf{Gender Interactions (treatment × female\_baseline):}
                    \begin{itemize}
                        \item \textbf{Total years enrolled:} $\beta = -0.064$ 
                            (SE = 0.218, p = 0.768)
                        \item \textbf{Passed primary exam:} $\beta = 0.001$ 
                            (SE = 0.040, p = 0.976)
                        \item \textbf{Total hours worked:} $\beta = -3.980$ 
                            (SE = 2.007, p = 0.047**)
                    \end{itemize}

                    \textbf{BMI Interactions (treatment × BMI):}
                    \begin{itemize}
                        \item \textbf{Total years enrolled:} $\beta = -0.006$ 
                            (SE = 0.005, p = 0.197)
                        \item \textbf{Passed primary exam:} $\beta = -0.001$ 
                            (SE = 0.001, p = 0.316)
                        \item \textbf{Total hours worked:} $\beta = 0.075$ 
                            (SE = 0.039, p = 0.054*)
                    \end{itemize}
                    
                    Note: * p $<$ 0.10, ** p $<$ 0.05, *** p $<$ 0.01

                \item[(ii)] Based on the regression results, the following 
                    interactions are significant at the 10\% level:

                    \begin{itemize}
                        \item \textbf{Total hours worked × Female (Model 8):} 
                            The interaction term is $-3.98$ (SE = 2.01, 
                            p = 0.047), which is statistically significant at 
                            the 5\% level. This indicates that the deworming 
                            treatment had a differential impact on hours worked 
                            by gender. For males, the treatment increased hours 
                            worked by 3.51 hours per week, but for females, the 
                            treatment effect was essentially zero (3.51 - 3.98 
                            = -0.47 hours). This suggests that males benefited 
                            more from the treatment in terms of increased labor 
                            market participation.
                        
                        \item \textbf{Total hours worked × BMI (Model 11):} The 
                            interaction term is 0.075 (SE = 0.039, p = 0.054), 
                            which is marginally significant at the 10\% level. 
                            This positive interaction coefficient suggests that 
                            the treatment effect on hours worked increases with 
                            BMI. In other words, students with higher BMI at 
                            baseline (healthier) experienced a larger increase 
                            in hours worked from the deworming treatment 
                            compared to students with lower BMI. For each 
                            one-unit increase in baseline BMI, the treatment 
                            effect on hours worked increases by 0.075 hours 
                            per week.
                    \end{itemize}

                    No other interactions were significant at the 10\% level:
                    \begin{itemize}
                        \item Total years enrolled × Female: p = 0.768
                        \item Passed primary exam × Female: p = 0.976
                        \item Total years enrolled × BMI: p = 0.197
                        \item Passed primary exam × BMI: p = 0.316
                    \end{itemize}

            \end{enumerate}
                
        \end{enumerate}

    \end{solution}

\end{homeworkProblem}