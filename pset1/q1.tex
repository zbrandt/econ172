\begin{homeworkProblem}{Short questions}

    \begin{enumerate}
        \item What is the poverty gap? What is the drawback of poverty head 
        count ratio? What is the advantage of poverty gap over the headcount 
        ratio?
        \item Consider a country in which 20 percent of the population live on 
        20 cents per day, 15 percent live on 30 cents per day, 15 percent live 
        on 50 cents per day, 10 percent live on 70 cents per day, 20 percent 
        live on 90 cents per day, and 20 percent live on 1.50 dollars per day. 
        If the poverty line is set at one dollar per day, what is the poverty 
        headcount ratio? What is the poverty gap? What is the poverty severity?
        \item Show an example that $P_1$ (poverty gap) and $P_2$ (poverty 
        severity) can change while $P_0$ (headcount ratio) is fixed. (You may 
        not need to calculate each value)
        \item How $P_0$ (headcount ratio), $P_1$ (poverty gap), $P_2$ (poverty 
        severity) change in graph (a), (b), (c) between situation 1 (black 
        line) and situation 2 (blue line)?
'    \end{enumerate}

    \begin{solution}
    
        \begin{enumerate}
            \item The poverty gap measures the intensity of poverty by 
                calculating the mean shortfall of the total population from 
                the poverty line, expressed as a ratio with the poverty line. 
                The drawback of the headcount ratio, the percentage of the 
                population living below the poverty line, is that it does not 
                account for the intensity of poverty among those who are below 
                the poverty line, only the incidence of it. The advantage of 
                the poverty gap over the headcount ratio is that it provides 
                a more comprehensive measure of poverty by considering both 
                the incidence and intensity of poverty.
            \item The poverty headcount ratio is 0.8. The poverty gap is 
                \[
                    \begin{aligned}
                        P_1 &= \frac{1}{N} \sum_{i=1}^{q} \left( 
                            \frac{z - y_i}{z} \right) = \frac{1}{100} \Bigg(
                            20 \times \frac{1.00 - 0.20}{1.00} + 15 \times 
                            \frac{1.00 - 0.30}{1.00} \\
                            &+ 15 \times \frac{1.00 - 0.50}{1.00} + 10 \times 
                            \frac{1.00 - 0.70}{1.00} + 20 
                            \times \frac{1.00 - 0.90}{1.00} \Bigg) \\
                            &= 0.39.
                    \end{aligned}
                \]
                The poverty severity is 
                \[
                    \begin{aligned}
                        P_2 &= \frac{1}{N} \sum_{i=1}^{q} \left( 
                            \frac{z - y_i}{z} \right)^2 = \frac{1}{100} \Bigg(
                            20 \times \left( \frac{1.00 - 0.20}{1.00} \right)^2
                            + 15 \times \left( \frac{1.00 - 0.30}{1.00} \right)^2 \\
                            &+ 15 \times \left( \frac{1.00 - 0.50}{1.00} \right)^2
                            + 10 \times \left( \frac{1.00 - 0.70}{1.00} \right)^2
                            + 20 \times \left( \frac{1.00 - 0.90}{1.00} \right)^2
                        \Bigg) \\
                        &= 0.25.
                    \end{aligned}
                \]
            \item Consider a country with 100 people and a poverty line of 
                \$1.00. Situation 1: 20 people live on \$0.50, 10 people live 
                on \$0.75, and 70 people live on \$1.50. Situation 2: 20 people 
                live on \$0.25, 10 people live on \$0.75, and 70 people live on 
                \$1.50. In both situations, the headcount ratio ($P_0$) is 30\% 
                (30 people are below the poverty line). However, the poverty 
                gap ($P_1$) and poverty severity ($P_2$) change due to the 
                different income levels of the poor.
            \item In graph (a), the blue line lies above the black line at all
                fractions of households and rises above the poverty line earlier
                as well. This indicates that the headcount ratio ($P_0$) has
                decreased, as fewer households are below the poverty line. The
                poverty gap ($P_1$) has also decreased, as the area between the
                blue line and the poverty line is smaller than that between the
                black line and the poverty line. The poverty severity ($P_2$)
                has decreased as well, as the blue line is closer to the poverty
                line than the black line, indicating that the poorest households
                are less poor in situation 2 compared to situation 1.

                In graph (b), the blue line intersects the poverty line at the
                same point as the black line, but is higher than the black line
                at all fractions of households before this point. Thereafter, it
                follows the same path as the black line. This indicates that the
                headcount ratio ($P_0$) remains unchanged, as the same number of
                households are below the poverty line. However, the poverty gap
                ($P_1$) has decreased, as the area between the blue line and the
                poverty line is smaller than that between the black line and the
                poverty line. The poverty severity ($P_2$) has also decreased,
                as the blue line is closer to the poverty line than the black
                line, indicating that the poorest households are less poor in
                situation 2 compared to situation 1.

                In graph (c), the areas of A and B are equal, and the poverty
                gap ($P_1$) and headcount ratio ($P_0$) remain unchanged. For 
                the poverty gap, since the areas of A and B are equal, the sum
                of the shortfalls from the poverty line remains the same.
                However, the blue line bumps up earlier than the black line so
                the poverty severity ($P_2$) is lower since the most poor are
                lifted up, reducing the severity measure.
        \end{enumerate}

    \end{solution}

\end{homeworkProblem}