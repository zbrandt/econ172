\begin{homeworkProblem}{Regression}
    
    A group of researchers wishes to test whether cash transfers to low-income 
    households have any spillovers on local small business owners. Their 
    sample consists of 160 villages across 20 districts in India. The 
    researchers randomly assign 80 villages to control and 80 villages to 
    treatment, stratifying by district. In control villages, no households 
    receive cash transfers. In treatment villages, approximately one fourth 
    of households receive a cash transfer. \\

    The researchers conduct surveys with a sample of 10–15 small business 
    owners in each village 3 months after the cash transfers are disbursed. 
    Note that none of the small business owners are also cash transfer 
    recipients. In these surveys, the researchers collect information such 
    as the nature of the business as well as monthly revenues, labor costs, 
    input costs, and profits. The researchers run the following regression:

    \[
        Y_{ivd} = \alpha_0 + \alpha_1 \text{Treatment}_{v} + \varepsilon_{ivd},
    \]

    where $Y_{ivd}$ represents outcomes of interest (revenues, profits) for 
    business owner $i$ in village $v$ in district $d$, and $\text{Treatment}_v$ 
    denotes treatment status at the village level. \\

    Here is the regression table and results (standard errors in parentheses). 
    The ``Village wealth index'' (VWI) combines several indicators of village 
    wealth, constructed so the index has mean zero and standard deviation one. 
    Positive values indicate a wealthier village, and the index is measured in 
    standard deviation units. ``Above median business assets'' is a dummy 
    variable equal to 1 if the business has above median assets and 0 
    otherwise.

    \begin{itemize}
        \item[(a)] Write out the regression equation for results presented in 
        column (3). What’s the variable of interest and coefficient?
        \item[(b)] Columns (1) and (2) above show the results of this analysis. 
        Interpret these results, commenting on the magnitude and significance 
        of the parameters of interest.
        \item[(c)] What is the 95\% confidence interval around the estimated 
        treatment effect in column (1) and (2)? What does this confidence 
        interval tell you?
        \item[(d)] In columns (3) and (4), the researchers test whether the 
        cash transfers had differential impacts on business revenues based on 
        individual business owner’s characteristics and village characteristics. 
        Write down the regressions depicted in columns (3) and (4). Interpret 
        each of the parameters in detail.
        \item[(e)] How would you summarize these results?
    \end{itemize}

    \begin{solution}
        
        \begin{itemize}
            \item[(a)] The regression equation for column (3) is:
                \[
                    Y_{ivd} = \alpha_0 + \alpha_1 \text{Treatment}_{v} + 
                    \alpha_2 \text{Village wealth index}_{iv} + 
                    \alpha_3 \left( \text{Treatment}_{v} \times 
                    \text{VWI}_{iv} \right) + \varepsilon_{ivd}
                \]
                The variable of interest is $\text{Treatment}_v$, and the
                coefficient is $\alpha_1 = 280.1$.
            \item[(b)] Column (1) shows that the treatment effect on monthly
                revenues is significant at the 1\% level, with a coefficient
                of 180.3. This indicates that, on average, business owners in
                treatment villages have monthly revenues that are \$180.3
                higher than those in control villages. Column (2) shows that
                the treatment effect on monthly profits is only significant at
                the 10\% level, with a coefficient of 22.5. This suggests that
                business owners in treatment villages have monthly profits that
                are \$22.5 higher than those in control villages, but the effect
                is less robust than for revenues.
            \item[(c)] The 95\% confidence interval for the treatment effect in
                column (1) is approximately (65.4, 295.2), calculated as
                $180.3 \pm 1.96 \times 58.6$. The 95\% confidence interval for
                the treatment effect in column (2) is approximately (–1.8, 46.8),
                calculated as $22.5 \pm 1.96 \times 12.4$. For the second 
                confidence interval, since it includes zero, we cannot be confident
                that the treatment effect is different from zero at the 95\% level.
            \item[(d)] The regression equation for column (3) is:
                \[
                    Y_{ivd} = \alpha_0 + \alpha_1 \text{Treatment}_{v} + 
                    \alpha_2 \text{Village wealth index}_{iv} + 
                    \alpha_3 \left( \text{Treatment}_{v} \times 
                    \text{VWI}_{iv} \right) + \varepsilon_{ivd}.
                \]
                $\alpha_1 = 280.1$ represents the treatment effect on
                monthly revenues for business owners in villages with an
                average wealth index (VWI = 0). $\alpha_2 = 130.8$ indicates
                that for each one standard deviation increase in the village
                wealth index, monthly revenues increase by \$130.8 on average,
                holding treatment status constant. $\alpha_3 = -105.3$ suggests
                that the treatment effect decreases by \$105.3 for each one
                standard deviation increase in the village wealth index. This
                implies that the treatment effect is larger in poorer villages
                (lower VWI) and smaller in wealthier villages (higher VWI). \\

                The regression equation for column (4) is:
                \[
                    Y_{ivd} = \alpha_0 + \alpha_1 \text{Treatment}_{v} + 
                    \alpha_2 \text{Above median business assets}_{iv} +
                    \alpha_3 \left( \text{Treatment}_{v} \times 
                    \text{Above-median}_{iv} \right) + 
                    \varepsilon_{ivd}.
                \]
                $\alpha_1 = 240.4$ represents the treatment effect on
                monthly revenues for business owners in villages with above-median
                business assets. $\alpha_2 = 110.5$ indicates that for each one
                standard deviation increase in the above-median business assets,
                monthly revenues increase by \$130.8 on average, holding treatment
                status constant. $\alpha_3 = -68.3$ suggests that the treatment
                effect decreases by \$68.3 for each one standard deviation increase
                in the above-median business assets. This implies that the treatment
                effect is larger in villages with lower above-median business assets
                and smaller in villages with higher above-median business assets.

        \end{itemize}

    \end{solution}

\end{homeworkProblem}